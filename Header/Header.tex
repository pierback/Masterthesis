%allgemeine Formatangaben
\documentclass[
 a4paper, 										% Papierformat
 12pt,												% Schriftgröße
 ngerman, 										% für Umlaute, Silbentrennung etc.
 titlepage,										% es wird eine Titelseite verwendet
 bibliography=totoc,					% Literaturverzeichnis im Inhaltsverzeichnis aufführen
 listof=totoc,								% Verzeichnisse im Inhaltsverzeichnis aufführen
 twoside, 										% einseitiges Dokument
 captions=nooneline,					% einzeilige Gleitobjekttitel ohne Sonderbehandlung wie mehrzeilige Gleitobjekttitel behandeln
 numbers=noenddot,						% Überschriften-??Nummerierung ohne Punkt am Ende
 parskip=half,							% zwischen Absätzen wird eine halbe Zeile eingefügt
 ]{scrbook}

\usepackage[bindingoffset=10mm]{geometry}% Bindeverlust von 8mm einbeziehen

\newcommand{\titel}{Learning drinking patterns with Q-Learning and Feature Selection on an Ethereum Blockchain}
 			% einbinden von persönlichen Daten

% Anpassung an Landessprache
\usepackage[ngerman]{babel}
 
% Verwenden von Sonderzeichen und Silbentrennung
%\usepackage[latin1]{inputenc}	
%\usepackage[T1]{fontenc}
%\usepackage[ansinew]{inputenc}
%\usepackage[applemac]{inputenc}
\usepackage[utf8]{inputenc}
\usepackage[T1]{fontenc} 
\usepackage{lmodern}			
\usepackage{textcomp} 																% Euro-Zeichen und andere
\usepackage[babel,german=quotes]{csquotes}						% Anführungszeichen
\RequirePackage[ngerman=ngerman-x-latest]{hyphsubst} 	% erweiterte Silbentrennung

% Befehle aus AMSTeX für mathematische Symbole z.B. \boldsymbol \mathbb
\usepackage{amsmath,amsfonts}

% Zeilenabstände und Seitenränder 
\usepackage{setspace}
\usepackage{geometry}

% Einbinden von JPG-Grafiken
\usepackage{graphicx}

% zum Umfließen von Bildern
% Verwendung unter http://de.wikibooks.org/wiki/LaTeX-Kompendium:_Baukastensystem#textumflossene_Bilder
\usepackage{floatflt}

% Verwendung von vordefinierten Farbnamen zur Colorierung
% Palette und Verwendung unter http://kitt.cl.uzh.ch/kitt/CLinZ.CH/src/Kurse/archiv/LaTeX-Kurs-Farben.pdf
\usepackage[usenames,dvipsnames]{color} 

% Tabellen
\usepackage{array}
\usepackage{longtable}
\usepackage{booktabs}
\usepackage{multirow}

% einfache Grafiken im Code
% Einführung unter http://www.math.uni-rostock.de/~dittmer/bsp/pstricks-bsp.pdf
\usepackage{pstricks}

% Quellcodeansichten
\usepackage{verbatim}
\usepackage{moreverb} 											% für erweiterte Optionen der verbatim Umgebung
% Befehle und Beispiele unter http://www.ctex.org/documents/packages/verbatim/moreverb.pdf
%\usepackage{listings} 											% für angepasste Quellcodeansichten siehe
% Kurzeinführung unter http://blog.robert-kummer.de/2006/04/latex-quellcode-listing.html

% Glossar und Abbildungsverzeichnis
\usepackage[
nomain,
nonumberlist, %keine Seitenzahlen anzeigen
acronym,      %ein Abkürzungsverzeichnis erstellen
toc          %Einträge im Inhaltsverzeichnis
]      %im Inhaltsverzeichnis auf section-Ebene erscheinen
{glossaries}

% verlinktes und Farblich angepasstes Inhaltsverzeichnis
\usepackage[pdftex,
colorlinks=false,
linkcolor=InterneLinkfarbe,
urlcolor=ExterneLinkfarbe]{hyperref}
\usepackage[all]{hypcap}

% URL verlinken, lange URLs umbrechen
\usepackage{url}

% sorgt dafür, dass Leerzeichen hinter parameterlosen Makros nicht als Makroendezeichen interpretiert werden
\usepackage{xspace}

% Beschriftungen für Abbildungen und Tabellen
\usepackage{caption}

% Entwicklerwarnmeldungen entfernen
\usepackage{scrhack}

% Abstand zwischen /item verringern
\usepackage{mdwlist}

% Naturwissenschaftliche Zitierweise
\usepackage{natbib}

% Subfigures
\usepackage{subfigure}

\usepackage{algpseudocode}
\usepackage[Algorithmus]{algorithm}

\usepackage{enumitem}

\usepackage{grffile}

\usepackage[section]{placeins}

\newcommand{\nlparagraph}[1]{\paragraph{#1}\mbox{}\\}
\newcommand{\quotes}[1]{``#1''}					% einbinden der verwendeten Latex-Pakete


\onehalfspacing 							% 1,5facher Zeilenabstand

\definecolor{InterneLinkfarbe}{rgb}{0.1,0.1,0.3} 	% Farbliche Absetzung von externen Links
\definecolor{ExterneLinkfarbe}{rgb}{0.1,0.1,0.7}	% Farbliche Absetzung von internen Links
%\definecolor{InterneLinkfarbe}{rgb}{0.0,0.0,0.0} 	% Farbliche Absetzung von externen Links
%\definecolor{ExterneLinkfarbe}{rgb}{0.0,0.0,0.0}	% Farbliche Absetzung von internen Links

% Einstellungen für Fußnoten:
\captionsetup{font=footnotesize,labelfont=sc,singlelinecheck=true,margin={5mm,5mm}}

% Stil der Quellenangabe
\bibliographystyle{alpha}

%Ausschluss von Schusterjungen
\clubpenalty = 10000
%Ausschluss von Hurenkindern
\widowpenalty = 10000

% Befehle, die Umlaute ausgeben, führen zu Fehlern, wenn sie hyperref als Optionen übergeben werden
\hypersetup{
%    pdftitle={\titel \untertitel},
%    pdfauthor={\autor},
%    pdfcreator={\autor},
%    pdfsubject={\titel \untertitel},
%    pdfkeywords={\titel \untertitel},
}

% Beispiel für eine Listings-Codeumbebungen
% Bei mehreren Definitionen empfielt sich das auslagern in eine externe Datei
%\lstloadlanguages{Java,HTML}
%\lstset{
%	frame=tb,
%	framesep=5pt,
%	basicstyle=\footnotesize\ttfamily,
%	showstringspaces=false,
%	keywordstyle=\ttfamily\bfseries\color{CadetBlue},
%	identifierstyle=\ttfamily,
%	stringstyle=\ttfamily\color{OliveGreen},
%	commentstyle=\color{GrayBlue},
%	rulecolor=\color{Gray},
%	xleftmargin=5pt,
%	xrightmargin=5pt,
%	aboveskip=\bigskipamount,
%	belowskip=\bigskipamount
%} 

%Den Punkt am Ende jeder Beschreibung deaktivieren
\renewcommand*{\glspostdescription}{}

%Glossar-Befehle anschalten
\makeglossaries
\glsenablehyper
%Befehle für Abkürzungen
%\newacronym{MS}{MS}{Microsoft}
%Eine Abkürzung mit Glossareintrag
\newacronym{SC}{SC}{Smart Camera\protect\glsadd{glos:SC}}
\newacronym{UH}{UH}{uhhhh weia}

\glsaddall







%Befehle für Glossar
\newglossaryentry{glos:AD}{
name=Active Directory,
description={Active Directory ist in einem Windows 2000/" "Windows
Server 2003-Netzwerk der Verzeichnisdienst, der die zentrale
Organisation und Verwaltung aller Netzwerkressourcen erlaubt. Es
ermöglicht den Benutzern über eine einzige zentrale Anmeldung den
Zugriff auf alle Ressourcen und den Administratoren die zentral
organisierte Verwaltung, transparent von der Netzwerktopologie und
den eingesetzten Netzwerkprotokollen. Das dafür benötigte
Betriebssystem ist entweder Windows 2000 Server oder
Windows Server 2003, welches auf dem zentralen
Domänencontroller installiert wird. Dieser hält alle Daten des
Active Directory vor, wie z.B. Benutzernamen und
Kennwörter.}
}
\newglossaryentry{glos:AntwD}{name=Antwortdatei, description={Informationen zum
Installieren einer Anwendung oder des Betriebssystems.}}
