\chapter{Abstract}
In dieser Masterarbeit wird ein System vorgestellt, welches die Blockchain Technologie und Maschinelles Lernen in Einklang bringt. Das Ziel der Arbeit ist es zu erforschen, zu welchem Grad sich diese neuartigen Technologien bereits verbinden lassen. \\
Zur Abbildung dieser Forschungsfrage wurde eine Problemstellung formuliert, welche das Erlernen des Kaffeetrinkverhaltens eines Users anhand von Daten einer Blockchain beinhaltet.\\
Dazu wurde eine Systemarchitektur entworfen, in welcher Getränkedaten von einer Mobile-App (Tablet) auf eine private Blockchain geschrieben werden und anhand dessen eine Lerner-Instanz, unter Beobachtung der Blockchain, das Kaffeetrinkverhalten der einzelnen Nutzer erlernen kann.\\
Zur Beantwortung der Forschungsfrage wurde eine Studie am Lehrstuhl durchgeführt, bei welcher eruiert wurde inwiefern das Erlernen des Kaffeetrinkverhaltens durch Daten von der Blockchain möglich ist. Aufgrund veralteter Hardware und den daraus resultierenden Problemen und Einschränkungen, reduzierte sich die Laufzeit der Testphase von 2 Monaten auf 3,5 Wochen.\\
Als Grundlage für die Analyse und Auswertung der Ergebnisse, wurde eine Anwendung zur Simulation eines Users entwickelt. Mit dieser Anwendung erfolgte der Nachweis der Funktionalität des implementierten Reinforcement Learning Algorithmus. Zudem dienten die Ergebnisse der simulierten Durchläufe als Anhaltspunkt und Bewertungsgrundlage für die Resultate der Studie. \\
Schlussendlich konnten Similaritäten in den Ergebnissätzen der Studie und der Simulation aufgezeigt werden und die Forschungsfrage anhand von Tendenzen positiv beantwortet werden.
Da die Dauer der Studie einer der limitierenden Umstände der Arbeit war, 
wurde zur all­um­fas­senden Erörterung der Forschungsfrage, auf die Notwendigkeit einer längeren Studie, ohne die technisch beeinträchtigenden Faktoren, hingewiesen.