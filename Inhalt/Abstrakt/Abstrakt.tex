\chapter{Abstract}
In dieser Masterarbeit wird ein System vorgestellt, welches die Blockchain Technologie und Maschinelles Lernen in Einklang bringt. Dabei ist es das Ziel dieser Arbeit zu erforschen, zu welchem Grad sich diese neuartigen Technologien bereits verbinden lassen. \\
Zur Abbildung dieser Forschungsfrage wurde eine Problemstellung formuliert, welche das Erlernen des Kaffeetrinkverhaltens eines Users/einer Person anhand von Daten von der Blockchain beinhaltet.\\
Dazu wurde eine Systemarchitektur entworfen, in welcher Getränkedaten von einer Mobile-App (Tablet) auf eine private Blockchain geschrieben werden und aufgrund dessen eine Lerner-Instanz, unter Beobachtung der Blockchain, das Kaffeetrinkverhalten der einzelnen Nutzer erlernen kann.\\
Zur Beantwortung der Forschungsfrage wurde eine Studie am Lehrstuhl durchgeführt, bei welcher eruiert wurde inwiefern das Erlernen des Kaffeetrinkverhaltens durch von der Blockchain möglich ist. Aufgrund veralteter Hardware und den daraus resultierenden Problemen und Einschränkungen, reduzierte sich die Laufzeit der Testphase von 2 Monate auf 3,5 Wochen.\\
Als Grundlage für die Analyse und Auswertung der Ergebnisse, wurde eine Anwendung zur Simulation eines Users entwickelt. Mit dieser erfolgte der Nachweis der Funktionalität des implementierten Reinforcement Learning Algorithmus. Zudem dienten die Ergebnisse der simulierten Durchläufe als Anhaltspunkt und Bewertungsgrundlage für die Resultate der Studie. \\
Schlussendlich konnte, aufgrund der kurzen Laufzeit der Studie, trotz aufgezeigter Similaritäten in den Ergebnissätzen der Studie und der Simulation, die Forschungsfrage lediglich anhand von Tendenzen positiv beantwortet werden und auf die Notwendigkeit einer längeren Studie hingewiesen, um eine Beantwortung der Forschungsfrage in voller Gänze zu erhalten.
