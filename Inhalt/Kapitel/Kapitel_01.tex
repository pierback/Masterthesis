\chapter{Einleitung}
\label{kap:Kapitel01}
%\nocite{*} << blendet alle Referenzen ein, auch wenn diese nicht im Text zitiert wurden
%
       
%
\section{Aufbau dieser Arbeit}
%
 
%
\section{Problemstellung}
Im heutigen Zeitalter des Smartphones in dem die Überstimulation der Sinne durch Informationen von den unterschiedlichsten Kanälen (z.B. Social Media, Push-Notifications etc.) zum Alltag vieler gehört, ist es mittlerweile unabdingbar geworden neben Werbeanzeigen, auch Apps, Plattformen, Systeme zu personalisieren. Dabei soll dem Nutzer eine User-Experience geboten werden, welche ihn einerseits vor dieser Überstimulation durch irrelevante Informationen bewahrt und andererseits eine möglichst lange Interaktion mit der App, Website, Plattform gewährleistet. \\
Dahingehend ist der erste Schritt, wie auch bei den maßgeschneiderten Werbeanzeigen auf den Social-Media Plattformen, das Verhalten der User zu erlernen und anhand dieses Wissens App-Interfaces oder auch Hintergrundprozesse anzupassen, sodass für jeden Nutzer eine optimale User-Experience sichergestellt werden kann. Hierbei kann auch von sog. virtuellen Assistenten gesprochen werden, welche durch die direkte Interaktion mit dem User oder durch das Beobachten des Users, dessen Verhalten und Gewohnheiten versucht zu erlernen. \\
Die Problematik die diesen Anwendungen anhaftet, ist die Weitergabe der privaten Nutzerdaten an die Plattform- bzw. App-Server, auf welchen die Informationen abgespeichert und die Assistenten trainiert werden, da vor allem bei mobilen Endgeräten die Kapazitäten und die Rechenleistungen dafür nicht ausreichen. \\
Eine Möglichkeit diese Weitergabe zu vermeiden besteht in der Blockchain-Technologie. Hierbei werden die Daten nicht mehr zentral bei einem Knoten im Netzwerk abgespeichert, sondern dezentral in einer verteilten Datenbank, in der die Sicherheit der Daten und Privatsphäre der Nutzer gewährleistet ist. Aufgrund dieser Dezentralität können virtuelle Assistenten auch lokal gehostet und trainiert werden, ohne dabei die Daten nach außen geben zu müssen. Durch diesen Ansatz entstehen jedoch bestimmte Problemstellungen, für die es in der Form noch keine standardisierten Lösungen gibt und erst zu eruieren gilt. \\
So lautet die Grundsatzfrage: wie gut lässt sich Blockchain mit Machine Learning verbinden? Also ist möglich virtuelle Assistenten mit den Daten von einer Blockchain zu versorgen, um damit das Verhalten eines Users zu erlernen. 
Gerade im Bezug auf das Erlernen des Nutzerverhaltens ist dies ein noch sehr unerforschtes Gebiet. \\
Eine weitere Frage welche aus dieser Problemstellung resultiert ist: wie eine notwendige \quotes{Feature Selection} auf einer Blockchain aussieht?\\\\
Um dies zu erforschen und abzubilden, wird in dieser Arbeit ein System vorgestellt, welches sich den gerade eben geschilderten Problemstellungen annimmt. \\
Hierbei wird eine Getränkeliste an einem Lehrstuhl durch eine Tablet-App ersetzt, in welcher die konsumierten Getränke (Kaffee, Club Mate, Wasser) eingegeben werden. 
Die kontextuellen Daten werden daraufhin auf eine private Blockchain gespeichert und das Getränk zudem mit einem Blockchain basierten Token bezahlt. 
Ein lokal gehosteter Machine Learning Algorithmus liest die Informationen von der Blockchain aus und lernt für jeden Nutzer ein Modell, welches dessen Kaffeetrinkverhalten abbildet. 


 zum einen private Blockchain aufgesetzt, welche einerseits für 
Dabei soll ein lokal gehosteter Machine Learning Algorithmus 


\section{Related Work}
%

%

