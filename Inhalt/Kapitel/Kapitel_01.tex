\chapter{Einleitung}
\label{kap:Kapitel01}
%\nocite{*} << blendet alle Referenzen ein, auch wenn diese nicht im Text zitiert wurden
%
      
%
\section{Aufbau dieser Arbeit}
%
 
%
\section{Problemstellung}
Im heutigen Zeitalter des Smartphones in dem die Überstimulation \cite{SMPUse} der Sinne durch Informationen von den unterschiedlichsten Kanälen (z.B. Social Media, Push-Notifications etc.) zum Alltag vieler gehört, ist es mittlerweile unabdingbar geworden neben Werbeanzeigen, auch Apps, Systeme und Services zu personalisieren. Dabei soll dem Nutzer eine User-Experience geboten werden, welche ihn einerseits vor dieser Überstimulation durch irrelevante Informationen bewahrt und andererseits eine möglichst lange Interaktion mit der App, Website, Plattform gewährleistet. \\
Dahingehend ist der erste Schritt, wie auch bei den maßgeschneiderten Werbeanzeigen auf den Social-Media Plattformen, das Verhalten der User zu erlernen und anhand dieses Wissens App-Interfaces oder auch Hintergrundprozesse anzupassen, sodass für jeden Nutzer eine optimale User-Experience sichergestellt werden kann. Hierbei kann auch von sog. virtuellen Assistenten gesprochen werden, welche durch die direkte Interaktion mit dem User oder durch das Beobachten des Users, dessen Verhalten und Gewohnheiten versucht zu erlernen. \\
Die Problematik die diesen Anwendungen anhaftet, ist die Weitergabe der privaten Nutzerdaten an die Plattform- bzw. App-Server, auf welchen die Informationen abgespeichert und die Assistenten trainiert werden, da vor allem bei mobilen Endgeräten die Kapazitäten und die Rechenleistungen dafür nicht ausreichen. \\
Eine Möglichkeit diese Weitergabe zu vermeiden besteht in der Blockchain-Technologie. Hierbei werden die Daten nicht mehr zentral bei einem Knoten im Netzwerk abgespeichert, sondern dezentral in einer verteilten Datenbank, in der die Sicherheit der Daten und Privatsphäre der Nutzer gewährleistet ist. Aufgrund dieser Dezentralität können virtuelle Assistenten auch lokal gehostet und trainiert werden, ohne dabei die Daten nach außen geben zu müssen. Durch diesen Ansatz entstehen jedoch bestimmte Problemstellungen, für die es in der Form noch keine (standardisierten) Lösungsansätze gibt und erst zu eruieren gilt. \\
So lautet die Forschungsfrage: \quotes{Inwiefern lassen sich Machine Learning und die Blockchain Technologie verbinden bzw. wie gut lässt sich dies bereits umsetzen?} \\
Beziehungsweise konkret auf einen Anwendungsfall bezogen formuliert: \quotes{Ist es möglich einen Machine Learning Algorithmus anhand von Daten von einer Blockchain zu trainieren, um damit das Verhalten eines Users zu erlernen?}
Gerade im Bezug auf das Erlernen des Nutzerverhaltens ist dies ein noch sehr unerforschtes Gebiet. \\
Um dies zu erforschen und abzubilden, wird in dieser Arbeit ein System vorgestellt, welches sich den gerade eben geschilderten Fragen annimmt. \\
Hierbei wird eine Getränkeliste an einem Lehrstuhl durch eine Tablet-App ersetzt, in welcher die konsumierten Getränke (Kaffee, Club Mate, Wasser) eingegeben werden. 
Die kontextuellen Daten werden daraufhin auf eine private Blockchain gespeichert und das Getränk zudem mit einem Blockchain basierten Token bezahlt. 
Ein lokal gehosteter Reinforcement Learning Algorithmus liest die Informationen von der Blockchain aus und lernt für jeden Nutzer ein Modell, welches dessen Kaffeetrinkverhalten abbildet. Mit dem Fokus auf das Kaffeetrinkverhalten, soll einerseits die Komplexität des Lernproblems reduziert und andererseits die Durchführung einer problemspezifischen \quotes{Feature Selection} ermöglicht werden. \\
Das System wird am Lehrstuhl installiert und im Zeitrahmen von 2 Monaten getestet. In diesem Zeitraum soll eruiert werden, inwiefern es der Learner-Instanz möglich ist, das jeweilige Kaffeetrinkverhalten eines Lehrstuhlmitarbeiters zu erlernen. \\
Außerdem wird eine Anwendung zur Simulation eines realen Users präsentiert, deren Zweckmäßigkeit darin besteht, die Funktionstüchtigkeit des implementierten Algorithmus bezüglich dessen Lernfähigkeit zu bezeugen. Dabei werden die Daten dem Lernalgorithmus direkt übergeben, ohne diese zuvor auf die Blockchain zu schreiben. \\
Zudem werden die Ergebnisse der Simulation analysiert und bei der Auswertung der Resultate aus der Testphase bzw. Studie in Bezug genommen. 

\section{Related Work}
In diesem Abschnitt werden andere Arbeiten aufgeführt, welche in der Thematik Ähnlichkeit in einen oder mehreren Punkten der Problemstellung bzw. Lösungsansätzen aufweisen. \\
Es sei darauf hingewiesen, dass, aufgrund der speziellen Problemstellung und der Neuartigkeit der Technologien, lediglich zwei Arbeiten gefunden wurden, welche jeweils in den Punkten Blockchain und Machine Learning an Similaritäten verfügen. \\
Zaidenberg et. al. \cite{zaidenberg:hal-00788028} beschreiben einen virtuellen (Desktop-) Assistenten, welcher in der Al­go­rith­mik und Vorgehensweise zu der in dieser Arbeit sehr ähnlich ist. Der Anspruch dieses virtuellen Assistenten ist die Unterstützung eines Users bei seinen täglichen Aufgaben. So soll dieser z.B. sicherstellen, dass der Nutzer keine Termine verpasst oder  nicht durch Benachrichtigungen abgelenkt wird (Email Programm wird vom Assistenten geschlossen), sollte er sich gerade in einem Meeting befinden. \\
Das ist Ziel die Reduktion der kognitiven Belastung des Users durch alltägliche Aufgaben. \\
Die Herausforderung des Systems besteht in der automatischen Ak­qui­rie­rung der User Präferenzen bzw. dem Erlernen des Userverhaltens und ist somit der Problemstellung dieser Arbeit äußerst ähnlich.\\
Als Algorithmus zum Erlernen des Userverhaltens wird auch hier eine Form des Q-Learning verwendet (DYNA-Q) \cite{Sutton}, welches im Gegensatz zum modellfreien Ansatz des Q-Learning, zusätzlich ein Modell der Umgebung konstruiert, um die Dauer der Lernphase zu verkürzen.\\
Die Unterschiede zur Systemarchitektur dieser Arbeit, liegen zum einen in der Vakanz einer Blockchain Komponente und zum anderen in der stärkeren Präsenz des Assistenten, welcher in einer weitaus ausgeprägteren Form mit dem User interagiert und auf ein direkteres Feedback angewiesen ist. \\\\
Singla et. al \cite{Singla:MLSDPUB} schildern ein Blockchain (Ethereum) basiertes System, um Userverhalten bzw. Useraktivität vorherzusehen.\\
Konkret wird eine Personalisierung von Smart Home Geräten durchgeführt, indem individuelle User-Geräte-Profile erstellt und dezentral (IPFS) abgespeichert werden. Betritt ein User einen Raum, wird dieser vom System erkannt und die entsprechenden Profile für jedes Gerät im Raum geladen. Diese Profile werden jedoch nicht direkt auf der Blockchain abgespeichert, sondern nur deren Hashwerte. Wobei jeder Hashwert bzw. jedes Profil einem Block auf der Blockchain entspricht. Der eigentliche Speicherort der Profile befindet sich auf dem IPFS (Interplanetary File System). IPFS ist eine verteilte Datenbank bzw. ein verteiltes Dateisystem mit einer Peer-to-peer Netzwerktopologie, die es ermöglicht Daten dezentral zu speichern \cite{IPFSIO, IPFSWP}.\\
Somit dienen die auf der Blockchain gespeicherte Hashwerte als Key, um das richtige User-Geräte-Profil von IPFS zu laden. Der Austausch und Zugriff der Profile wird dabei von Smart Contracts übernommen.\\
Zum Erlernen der Useraktivität wird eine Assoziationsanalyse (association rule mining) verwendet. Dabei werden die Gerätelogs periodisch durchsucht und anhand der vordefinierten Geräte Profil Daten (Waschmaschine \= Wochentag, Zeit, Wassertemperatur, ...) die Userprofile erstellt. Eine Assoziationsanalyse funktioniert dahingehend, dass es anhand von Auftrittswahrscheinlichkeiten Korrelationen zwischen Eingabeparameter ermittelt und dadurch Muster in Datensätzen aufdeckt\cite{Cios2007, online:associationrules}.\\\\
Im Kontrast zur Problemstellung dieser Arbeit, liegt der Unterschied in der geringeren Komplexität des Lernproblems, weswegen die Ansätze bzw. Algorithmen gänzlich differieren, auf welche Art und Weise das Userverhalten gelernt wird.