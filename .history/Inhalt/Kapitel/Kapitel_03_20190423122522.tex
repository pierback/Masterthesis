\usepackage[utf8]{inputenc}
\chapter{System Architektur}
\label{kap:Kapitel03}

\section{\"Uberblick}
Im folgenden wird nun die System Architektur erläutert, welche als Grundlage 
der Studie diente, um die in 1.2 geschilderte Problemstellung abzubilden und 
letztendlich zu lösen. \hspace*{\fill}\linebreak[0]%
Zuerst widme ich mich dem Prozess der Entwicklung, aus welchem schließlich 
die finale Version der System Architektur resultierte. 
Die Entwicklungszeit betrug ca. 5 Monate und beinhaltete mehrere Iterationen 
der einzelnen Komponenten hin zum derzeitigen Stand. Das Konzept sah primär 
die Entwicklung von drei dedizierten Softwareanwendungen vor, welche aber 
im Zuge der Iterationen nochmal in kleinere Module aufgeteilt und 
ausgelagert wurden. Zudem wurden, um den Workflow und das Testen während 
der der Entwicklungsphase zu erleichtern, Anwendungen entwickelt, 
welche während Konzeption in der Art nicht vorgesehen waren, aber 
partiell Bestandteil der Systemarchitektur wurden.
\subsection{Architektur}
\subsection{Workflow}





\section{Blockchain}
\subsection{Beveragelist}
\subsection{ERC-Token}

\section{Q-Learning}
\subsection{Warum Q-Learning?}
\subsection{Modellierung}
\subsection{Lernprozess \& Ablauf}

\section{Mobile App}


