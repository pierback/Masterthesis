\chapter{System Architektur}
\label{kap:Kapitel03}

\section{Überblick}
Im folgenden wird nun die System Architektur erläutert, welche als Grundlage der Studie diente, um die in 1.2 geschilderte Problemstellung abzubilden und letztendlich zu lösen. 
Zuerst widme ich mich dem Prozess der Entwicklung, aus welchem schließlich die finale Version der Systemarchitektur resultierte. \\
Die Entwicklungszeit betrug ca. 5 Monate und beinhaltete mehrere Iterationen der einzelnen Komponenten hin zum derzeitigen Stand. Das Konzept sah primär die Entwicklung von drei dedizierten Softwareanwendungen vor, welche aber im Zuge der Iterationen nochmal in kleinere Module aufgeteilt und ausgelagert wurden. Zudem wurden, um den Workflow und das Testen während der Entwicklungsphase zu erleichtern, Anwendungen entwickelt, welche während Konzeption in der Art nicht vorgesehen waren, aber partiell Bestandteil der Systemarchitektur wurden. \\
So wurde mit zwei separaten Repos gestartet, einerseits für den Learning-Part, welcher anfänglich auch die Smart Contracts umfasste, und andererseits eines für die Mobile-App, welches bereits vor der eigentlichen Konzeption erstellt wurde, um in erster Linie bestehende Crossplattform Frameworks, auf Basis der Kompatibilität und Funktionstüchtigkeit mit Libaries, welche die Kommunikation mit der Blockchain ermöglichen, zu evaluieren. Die Wahl fiel letztendlich auf React-Native, welches zwar nur bis zu einer bestimmten Versionsnummer mit der web3.js Libary von Ethereum vollends kompatibel ist und nur mit einem kleinen Workaround zum Laufen gebracht werden konnte. Dies war jedoch mit anderen Frameworks (z.B. Nativescript) in keiner Weise zu bewerkstelligen, was v.a. der größeren Community und dem (forgeschrittenen) Alter (4 Jahre) von React-Native geschuldet ist \\
\subsection{Architektur}
\subsection{Workflow}





\section{Blockchain}
\subsection{Beveragelist}
\subsection{ERC-Token}

\section{Q-Learning}
\subsection{Warum Q-Learning?}
\subsection{Modellierung}
\subsection{Lernprozess \& Ablauf}

\section{Mobile App}


